%\textbf{Hinweise zum Titel der Abschlussarbeit:}
%Der Titel der Abschlussarbeit ist deren „Aushängeschild“ und daher sehr wichtig. Er soll prägnant und verständlich formuliert sein. Folgende Fragen sollten helfen dies zu erreichen:
%\begin{itemize}
%\item Trifft der Titel den geplanten Inhalt der Arbeit? Ist der Titel kurz (maximal Zweizeiler), prägnant und allgemein verständlich?
%\item Wird nur eine Sprache verwendet (deutsch oder englisch und nicht denglisch)? Sind die verwendeten Abkürzungen allgemein bekannt oder können Sie nicht auch vermieden werden?
%\item Würden Sie sich selbst für die Arbeit nur aufgrund des gewählten Titels interessieren und diese lesen wollen?
%\item Würde ihr zukünftiger Arbeitgeber den Titel verstehen?
%\end{itemize}
%\textbf{Hinweise zum Inhalt des Exposè:}
%
%Das Exposé stellt die Grundlage für ein Arbeitsvorhaben dar, ist die Voraussetzung für die
%Anmeldung zur Abschlussarbeit. Das Exposé dient sowohl der eigenen Orientierung als auch der Verständigung zwischen
%Kandidatin / Kandidaten und Prüferin/ Prüfer. Auf ein bis zwei Seiten sollten zuerst folgende Punkte erläutert werden (Nicht alle Punkte
%sind bei jedem Thema relevant). Die nicht selbstständig getroffenen Aussagen sind mit Literaturquellen zu belegen.
\section{Problemstellung}
%\begin{itemize}
%\item Welches wissenschaftlich oder fachlich relevante Problem ist der Ausgangspunkt der Arbeit und warum handelt es sich dabei um ein Problem? 
%\item Hohe Quote für das Scheitern von IT Startups aufgrund von vermeidbaren Dingen.
%\item Welche Relevanz hat das untersuchte Problem?
%\item Warum ist es lohnenswert diesem Problem nachzugehen? Warum soll ausgerechnet dieses Problem behandelt werden?
%\item 
%\item Wie kommt es zu dem Thema?
%\end{itemize}

Ein Großteil aller IT-Startups scheitern unter Anderem aufgrund der fehlenden Nachfrage für ihr Produkt. Das hat zur Folge, dass sich immer weniger Leute trauen, überhaupt ein Unternehmen zu gründen. Sie wollen das finanzielle Risiko nicht eingehen und haben auch Angst, sich damit die Zukunft zu verbauen. Sehr viele heute sehr große Unternehmen sind aus einer kleinen simplen Idee heraus entstanden. Denn es genügt nicht, einfach nur eine Idee zu haben. Es ist auch wichtig, die Idee richtig umzusetzen und zu verkaufen, an den Kunden abzustimmen. Sonst ist auch die beste Idee nichts wert. Es stellt sich also die Frage, ob potentiell vermeidbare Probleme mithilfe von modernen und erfolgsversprechenden Prozessen eliminiert werden können. 

In der heutigen Welt hat jeder die Möglichkeit, sich selbstständig zu machen. Meist reicht ein kleines Team und eine gute Idee aus. Viele heute sehr große Unternehmen sind aus einem kleinen Startup entstanden und unglaublich erfolgreich geworden, wie zum Beispiel Facebook oder AirBnb. Dadurch ist für viele das Scheinbild entstanden, dass jede noch so einfache Idee zu einem Milliarden-Unternehmen werden könnte. 

Allerdings ist es statistisch erwiesen, dass rund 70\% aller IT-Startups scheitern. \citep{CBInsights_failure} Diese Statistik steht dem Scheinbild entgegen, nach welchen jede Idee Gold wert ist. Die Idee alleine genügt leider nicht, denn es ist auch essentiell, diese auf die Zielgruppe abzustimmen und richtig zu vermarkten. Die Gründe für das Scheitern der meisten Unternehmen sind oft sehr ähnlich. Einer der Hauptgründe dafür ist beispielsweise die fehlende Nachfrage. Außerdem war bei ungefähr 17\% aller gescheiterten Startups die schlechte Benutzerfreundlichkeit des Produktes ein Problem. Weitere Probleme stellen beispielsweise das falsche Team, fehlendes Budget, Konkurrenzkampf usw. dar. \citep{CBInsights_reasons}

Darüber hinaus ist es hauptsächlich in der IT-Branche ist es ein verbreiteter Fehler, dass sich die Entwickler sofort auf die Implementierung stürzen, ohne die Rahmenbedingungen zuerst genau abzustecken. Ein Entwickler denkt, er habe ein gutes Bild vom Endprodukt, ohne wirklich mit dem Kunden (oder der potentiellen Zielgruppe) geredet zu haben. Bei einem Startup gibt es zunächst noch keinen konkreten Kunden. Deshalb muss auf potentielle Kunden zugegangen werden, was auch oft weitestgehend vermieden wird. Am Ende wird das Produkt aus Entwicklersicht umgesetzt, was logischerweise in fehlender Nachfrage resultieren kann. Daraus könnte sich nun eine Kette von Problemen entwickeln. Da bei fehlendem Erfolg das Klima im Team mit Sicherheit zu leiden hat und auch die Sponsoren ausbleiben, ist ebenfalls ersichtlich. Entwickelt zeitgleich ein anderes Team ein ähnliches Produkt und schafft dieses eine bessere Umsetzung, wurde man klar von der Konkurrenz verdrängt.

\section{Fragestellung}
%\begin{itemize}
%\item Was genau werden Sie selbst untersuchen?
%\item Mit diesem Schritt soll das Thema weiter eingegrenzt werden.
%\item Auf welche zentrale Frage soll in der Arbeit eine Antwort gefunden oder gegeben werden?
%\item Welches konkrete Problem soll damit (aus welcher Perspektive und unter welchen Vorzeichen) behandelt werden? 
%\item Hier sollte eine Problemanalyse durchgeführt werden und Teilprobleme identifiziert werden.
%\end{itemize}

%In dieser Arbeit soll hauptsächlich der Prozess SPRINT von Googleventures an der Erstellung einer neuen Online-Plattform evaluiert werden. 
Um den oben beschriebenen Problemen entgegenzuwirken entstehen immer mehr Prozesse, mit welchen jede Idee vermarktbar und erfolgreich zu sein scheint. Die Gründer versprechen sehr großen Erfolg und auch in immer mehr Unternehmen werden diese Innovation-Prozesse für kleinere Verbesserungen oder auch größere Projekte angewandt. Diese bauen alle auf einem gleichen Prinzip auf: Das Produkt perfekt auf den Kunden abzustimmen.

Es stellt sich also die Frage, ob potentiell vermeidbare Probleme bei der Startup-Gründung mithilfe von modernen und erfolgsversprechenden Prozessen eliminiert werden können. 

\section{Ziele/ Hypothesen}
%\begin{itemize}
%\item Was soll mit den Ausführungen erreicht werden?
%\item Was soll belegt oder widerlegt werden?
%\item Beide Aspekte müssen mit den vorher aufgestellten (Leit-)Fragen übereinstimmen.
%\end{itemize}

Ziel dieser Arbeit soll es sein, eine Aussage darüber zu treffen, wie sinnvoll die Anwendung solcher Prozesse wirklich ist. Gibt es einen Mehrwert bei der Verwendung oder ähnelt das Produkt der initialen Idee des Entwicklers trotzdem sehr. Ist das Produkt besser? Welchen der Top-Gründe für das Scheitern eines Startups kann es eventuell noch entgegenwirken? Teambuilding, Budgetplanung, Rollenverteilung, Konkurrenzkampf.
Außerdem könnte erwartet werden, dass man durch die Kundennähe im Vorhinein schon eine große Liste an add-ons zusammenstellen kann, womit das Produkt immer wieder erweitert werden kann.

\section{Theoriebezug / Forschungsstand}
%\begin{itemize}
%\item Welche wissenschaftlichen Erkenntnisse liegen zu dem Thema bereits vor?
%\item Welche Aspekte des Themas sind bisher noch nicht ausreichend oder erfolgreich behandelt worden? 
%\item Auf welche Begriffe, Theorien, Modelle oder Erklärungsansätze soll Bezug genommen werden?
%\end{itemize}

Die Erfinder dieser Prozesse werben damit, dass damit jedes Startup Erfolg haben wird. Es gibt auch bereits sehr viele große Unternehmen, die diese Prozesse entweder für die Gründung oder für kleinere Erweiterungen benutzt haben. Entgegen stehen immer noch die Scheiterungsstatistiken von Startups. Daher könnte es auch einfach Zufall sein und jene Unternehmen hätten sich auch anders gut entwickelt. Es lässt sich leider keine Aussage darüber machen, wie viele von denen, die diese Prozesse angewandt haben, am Ende doch scheitern und warum. Kurz und knapp: Der Trend geht eindeutig in die Richtung, kreative Prozesse zu benutzen. Der Mehrwert wird hoch geschätzt, allerdings kommen diese Aussagen hauptsächlich von den Prozessgründern.

%\section{Methode}
%\begin{itemize}
%\item Mit welchen wissenschaftlichen Methoden soll das Problem bzw. Teilprobleme bearbeitet werden?
%\item Welche Methoden bieten sich an, die (Leit-)Fragen und Hypothesen angemessen zu
%bearbeiten? (theoretisch oder empirisch, qualitativ oder quantitativ, eine Kombination der
%Methoden, etc.).
%\end{itemize}

\section{Evaluierungsstrategie}
%\begin{itemize}
%\item Wie sollen die entwickelten Methoden evaluiert werden, so dass nachgewiesen werden
%kann, dass das / die Ziel(e) auch erreicht wurde(n).
%\end{itemize}

Das kann in diesem Fall natürlich nicht am Erfolg eines neu gegründeten Startups gemessen werden, da der auf mehrere Jahre hinweg gemessen wird. Es soll ein Vergleich stattfinden, was in etwa entstanden wäre ohne den Sprint und mit dem Sprint. Diese Ergebnisse sollen am Ende mit potentiellen Kunden abgeglichen werden, woran sich der Erfolg misst. Auch das Team selbst soll in die Evaluierung miteingebunden werden, ob diese den Prozess als gut oder überflüssig empfunden haben.
Weichen initiale Produktidee und Produktidee nach dem Sprint sehr weit voneinander ab, ist zu erwarten, dass eine klare Aussage getroffen wird, welches Produkt besser ist. Weichen sie nicht stark voneinander ab, ist der Erfolg von SPRINT an diesem Beispiel klar widerlegt. Jedoch ist auch hier zu erwarten, dass vom Team eine Aussage gemacht werden kann, ob der Prozess an sich sinnvoll ist und das nur Zufall/Glück war.






